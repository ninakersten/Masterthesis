\appendix

\chapter{Appendices}
\section{DREAM8-Challenge scoring priciples}

For determining the Receiver-Operating-Characteristic-Curve a True-Positive-Rate (TPR) and a False-Positive-Rate (FPR) is calculated.\\
\begin{defn}\textbf{True-Positive-Rate (TPR).}\\
\textit{The TPR values are for the y-axis of the ROC.}
\begin{equation}
TPR=\frac{TP}{TP+FN}
\end{equation}
\end{defn}
\begin{defn}
\textbf{False-Positive-Rate (FPR).}\\
\textit{The FPR values are for the x-axis of the ROC.}
\begin{equation}
FPR=\frac{FP}{FP+TN}
\end{equation}
\end{defn}

Threfore, a threshold $\tau$ is set and the confidence scores are classified by this threshold.  Confidence scores below this threshold take a value of $0$ indicating an edge is less likely to occur and a confidence score above $\tau$ is taking a value of $1$ indicating an edge is potential. By increasing the threshold $\tau\in[0,1]$ the amount false positve decrases and false negative increase. 
Resulting calasses are put into a context of True Positive Rate ($TPR=\frac{TP}{TP+FN}$) and False Negative Rate ($FPR=\frac{FP}{FP+TN}$). This yields a set of values returning a value of the area under the receiver operating characteristic curve (AUROC).\\\\   
Similar to balanced accuracy the AUPR (area under the precision recall curve) metric is used for imbalanced classes in the confusion matrix taking precision and recall in relationship by shifting $\tau$.

