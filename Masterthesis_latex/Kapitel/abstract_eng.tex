\thispagestyle{empty}


\vspace*{1cm}

\begin{center}
\textbf{Abstract}
\end{center}
The survival of a cell and eventually of its organism depends mostly on  the reliable interaction between different kinds of substances. Different functionalities inside and outside a cell like profileration, division and apoptosis are part of different regulatory networks in a system. Small malfunction in these regulatory networks could cause diseases from low impact for the organism to a big one. Therefore it is necessary to learn these regulatory networks, its structure and dynamical behaviour for further drug design approaches. Several efforts have been made to infer a regulatory network.%Quellen. 
In this work the focus is on inferring Boolean networks from real-life time-course data of breast cancer tissue, which provide a simplified version of the states of substances in a system (a gene is expressed:$1$, or not:$0$). The boolean network is validated for inferring the network by three inference different learning algorithms Reveal, Best-fit and Full-fit in combination with different k-means clustering binarization algorithms. The inferred networks are evaluated against a goldstandard to show how well the model predicted the networks structure. Thus, it is shown how the complexity and the size of a network influence the predictive power of the model and hence, the explanatory power of a boolean approach.

\vspace*{1cm}

\noindent 
%Give a short overview:
%Describe the Pipeline

