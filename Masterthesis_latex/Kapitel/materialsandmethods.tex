\chapter{Materials and Methods}
\section{Inferencealgorithms}
%Baysian networks,Boolean regulatory networks, Ordinary differential equation models and Neural networks
regression
\citep{CHAI201455}
%Welche Algorithmen gibt es?
%Warum wählen wir hier den Boolean Approach?
\subsection*{Boolean Approach}
%Define REVEAL, BESTFIT und FULLFIT 
\subsection*{Baysian networks}
\subsection*{Ordinary differential equation models}
\subsection*{Neural networks regression}
BIBN (Bayesian inference approach for a Boolean network)
REVEAL (Reverse Engineering algorithm)
PCA-CMI (Path consistency algorithm- Conditonal mutual information)
ARCANE (Time-delay algorithm for reconstruction of accurate cellular networks)
MIDER (Mutual information distance and entropy reduction)\citep{MIDER}
	defines a mutual information based distance between genes to specify the directionality
BESTFIT ()

These mutual information-based methods are computationally expensive, because they are implemented to compute exact mutual information values over all possible combinations of genes.

RelNet (Revelance network algorithm)

CLR (context likelihood of relatedness method)
CST (chi-square test)

\section{PyBoolNet}



\section{Data Selection}
\subsection{Example data set}
%How the example data set is created
%
\subsection{HPN-DREAM breast cancer data set}
Now it is shown how the Pipeline can be applied to a real-life time course data set. 
\subsubsection*{What is the Dream Challenge?}
For a Boolean network inference the data of a platform so-called Dialogue on Reverse Engineering Assessment and Methods (DREAM) - Challenge is used. The DREAM-Challenge is a non-profit, collaborative community effort consisting of contributors from across the research spectrum of questions in biology and medicine. This organization was built in 2006 and publishes crowdsourcing challenges with transparent sharing of data, thus everyone can participate the challenge. The DREAM-Challenge has partnered with Sage Bionetworks, which provide the infrastructure by Sage Bionetworks Synapse platform to get access to the open collaborative data analysis. Overall the DREAM-Challenge is a helpful instrument to get real-life data, comparing results and interact with other researchers all over the world, while contribute solutions to biological and medical questions.\citep{DreamChalleneg Homepage}.
The challenging question was to decide which Dream Challenge data set could be useful for this masterthesis. For inferring a Boolean network and further analysis of the state tarnsition graph the desired data set should contain measurements of experiments with less pertubational information and a in a time course context. 
The Dream Challenge 5 dealing with gene-gene interaction,providing test and training data sets of gene expressions seemed to be an appropriate candidate. But there was less time course information and a high abundance of pertubation. Thus the Dream8 Challenge is was chosen. This challenge describes protein-protein interaction and measurements for multiple timepoints.

\subsubsection{DREAM8}
%Short sentence about the DREAM8 Challenge: What is the goal(medically and mathematically)?What sub challenge do I do?

\subsubsection*{Data Collection}
The collection of the HPN8 breast cancer PPI data is done by a technique so called Reverse phase protein array(RPPA). This technique is divided up into 6 parts:
\subparagraph*{Sample collection}
%Wo kommen die breast cancer cells her?Erwähne: Es gibt 4 Zelllinien (BT20, UACC812, BT549, MCF7) und 8 Stimuli (Insulin, Serum, HGF, NRG1,EGF, FGF1,GF1,IGF1), welche am Ende 32 Netzwerke ergebe. Vielleicht die Zelllinien und die Stimulis erklären.
An inhibitor or stimulus in form of drugs is added to a set of celllines at the same time and the celllines are then processed at different time points.
\subparagraph*{Cell Lysis}
Cell fragments are lysed with a celllysis buffer to obtain high protein concentration.The choice of a buffer decides the quantity of proteins can be lysed out of the cell.
%Welche buffer wurden in der Dream Challenge verwendet.
\subparagraph*{Dilution}
Dilution of the celllysed probes.
\subparagraph*{Antibody screening}
The lysates are pooled and resolved by SDS-PAGE followed by western blotting on a nitrocellulose membrane. The membrane is cut into 4mmm strips. Each slide is probed with a  different antibody, primary with a secondary antibody.
\subparagraph*{Fluorometric detection}
Primary and secondary antibody are diluted.%with which buffer?TBST?%
Detection reagentisput on each slide. Signal amplification and detection is done by an optical flatbed scanner if colormetric technique is used orby laser scanning. %Welche methode wurde verwendet?
\subparagraph*{Data set structure}
Missing data points and outliers are detected and deleted from the data set. The data set is normalized %Welche Struktur weißt der Datensatz am Ende auf? Welche Informationen sind davon für uns wichtig? Normalisierungalgorithmus wurde hier verwendet?



